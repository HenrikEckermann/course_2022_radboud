% Options for packages loaded elsewhere
\PassOptionsToPackage{unicode}{hyperref}
\PassOptionsToPackage{hyphens}{url}
%
\documentclass[
  oneside]{book}
\usepackage{amsmath,amssymb}
\usepackage{lmodern}
\usepackage{iftex}
\ifPDFTeX
  \usepackage[T1]{fontenc}
  \usepackage[utf8]{inputenc}
  \usepackage{textcomp} % provide euro and other symbols
\else % if luatex or xetex
  \usepackage{unicode-math}
  \defaultfontfeatures{Scale=MatchLowercase}
  \defaultfontfeatures[\rmfamily]{Ligatures=TeX,Scale=1}
\fi
% Use upquote if available, for straight quotes in verbatim environments
\IfFileExists{upquote.sty}{\usepackage{upquote}}{}
\IfFileExists{microtype.sty}{% use microtype if available
  \usepackage[]{microtype}
  \UseMicrotypeSet[protrusion]{basicmath} % disable protrusion for tt fonts
}{}
\makeatletter
\@ifundefined{KOMAClassName}{% if non-KOMA class
  \IfFileExists{parskip.sty}{%
    \usepackage{parskip}
  }{% else
    \setlength{\parindent}{0pt}
    \setlength{\parskip}{6pt plus 2pt minus 1pt}}
}{% if KOMA class
  \KOMAoptions{parskip=half}}
\makeatother
\usepackage{xcolor}
\IfFileExists{xurl.sty}{\usepackage{xurl}}{} % add URL line breaks if available
\IfFileExists{bookmark.sty}{\usepackage{bookmark}}{\usepackage{hyperref}}
\hypersetup{
  pdftitle={Multi-omic data science with R/Bioconductor},
  hidelinks,
  pdfcreator={LaTeX via pandoc}}
\urlstyle{same} % disable monospaced font for URLs
\usepackage[top=30mm,left=15mm]{geometry}
\usepackage{color}
\usepackage{fancyvrb}
\newcommand{\VerbBar}{|}
\newcommand{\VERB}{\Verb[commandchars=\\\{\}]}
\DefineVerbatimEnvironment{Highlighting}{Verbatim}{commandchars=\\\{\}}
% Add ',fontsize=\small' for more characters per line
\usepackage{framed}
\definecolor{shadecolor}{RGB}{248,248,248}
\newenvironment{Shaded}{\begin{snugshade}}{\end{snugshade}}
\newcommand{\AlertTok}[1]{\textcolor[rgb]{0.94,0.16,0.16}{#1}}
\newcommand{\AnnotationTok}[1]{\textcolor[rgb]{0.56,0.35,0.01}{\textbf{\textit{#1}}}}
\newcommand{\AttributeTok}[1]{\textcolor[rgb]{0.77,0.63,0.00}{#1}}
\newcommand{\BaseNTok}[1]{\textcolor[rgb]{0.00,0.00,0.81}{#1}}
\newcommand{\BuiltInTok}[1]{#1}
\newcommand{\CharTok}[1]{\textcolor[rgb]{0.31,0.60,0.02}{#1}}
\newcommand{\CommentTok}[1]{\textcolor[rgb]{0.56,0.35,0.01}{\textit{#1}}}
\newcommand{\CommentVarTok}[1]{\textcolor[rgb]{0.56,0.35,0.01}{\textbf{\textit{#1}}}}
\newcommand{\ConstantTok}[1]{\textcolor[rgb]{0.00,0.00,0.00}{#1}}
\newcommand{\ControlFlowTok}[1]{\textcolor[rgb]{0.13,0.29,0.53}{\textbf{#1}}}
\newcommand{\DataTypeTok}[1]{\textcolor[rgb]{0.13,0.29,0.53}{#1}}
\newcommand{\DecValTok}[1]{\textcolor[rgb]{0.00,0.00,0.81}{#1}}
\newcommand{\DocumentationTok}[1]{\textcolor[rgb]{0.56,0.35,0.01}{\textbf{\textit{#1}}}}
\newcommand{\ErrorTok}[1]{\textcolor[rgb]{0.64,0.00,0.00}{\textbf{#1}}}
\newcommand{\ExtensionTok}[1]{#1}
\newcommand{\FloatTok}[1]{\textcolor[rgb]{0.00,0.00,0.81}{#1}}
\newcommand{\FunctionTok}[1]{\textcolor[rgb]{0.00,0.00,0.00}{#1}}
\newcommand{\ImportTok}[1]{#1}
\newcommand{\InformationTok}[1]{\textcolor[rgb]{0.56,0.35,0.01}{\textbf{\textit{#1}}}}
\newcommand{\KeywordTok}[1]{\textcolor[rgb]{0.13,0.29,0.53}{\textbf{#1}}}
\newcommand{\NormalTok}[1]{#1}
\newcommand{\OperatorTok}[1]{\textcolor[rgb]{0.81,0.36,0.00}{\textbf{#1}}}
\newcommand{\OtherTok}[1]{\textcolor[rgb]{0.56,0.35,0.01}{#1}}
\newcommand{\PreprocessorTok}[1]{\textcolor[rgb]{0.56,0.35,0.01}{\textit{#1}}}
\newcommand{\RegionMarkerTok}[1]{#1}
\newcommand{\SpecialCharTok}[1]{\textcolor[rgb]{0.00,0.00,0.00}{#1}}
\newcommand{\SpecialStringTok}[1]{\textcolor[rgb]{0.31,0.60,0.02}{#1}}
\newcommand{\StringTok}[1]{\textcolor[rgb]{0.31,0.60,0.02}{#1}}
\newcommand{\VariableTok}[1]{\textcolor[rgb]{0.00,0.00,0.00}{#1}}
\newcommand{\VerbatimStringTok}[1]{\textcolor[rgb]{0.31,0.60,0.02}{#1}}
\newcommand{\WarningTok}[1]{\textcolor[rgb]{0.56,0.35,0.01}{\textbf{\textit{#1}}}}
\usepackage{longtable,booktabs,array}
\usepackage{calc} % for calculating minipage widths
% Correct order of tables after \paragraph or \subparagraph
\usepackage{etoolbox}
\makeatletter
\patchcmd\longtable{\par}{\if@noskipsec\mbox{}\fi\par}{}{}
\makeatother
% Allow footnotes in longtable head/foot
\IfFileExists{footnotehyper.sty}{\usepackage{footnotehyper}}{\usepackage{footnote}}
\makesavenoteenv{longtable}
\usepackage{graphicx}
\makeatletter
\def\maxwidth{\ifdim\Gin@nat@width>\linewidth\linewidth\else\Gin@nat@width\fi}
\def\maxheight{\ifdim\Gin@nat@height>\textheight\textheight\else\Gin@nat@height\fi}
\makeatother
% Scale images if necessary, so that they will not overflow the page
% margins by default, and it is still possible to overwrite the defaults
% using explicit options in \includegraphics[width, height, ...]{}
\setkeys{Gin}{width=\maxwidth,height=\maxheight,keepaspectratio}
% Set default figure placement to htbp
\makeatletter
\def\fps@figure{htbp}
\makeatother
\setlength{\emergencystretch}{3em} % prevent overfull lines
\providecommand{\tightlist}{%
  \setlength{\itemsep}{0pt}\setlength{\parskip}{0pt}}
\setcounter{secnumdepth}{5}
\usepackage{booktabs}
\usepackage{booktabs}
\usepackage{longtable}
\usepackage{array}
\usepackage{multirow}
\usepackage{wrapfig}
\usepackage{float}
\usepackage{colortbl}
\usepackage{pdflscape}
\usepackage{tabu}
\usepackage{threeparttable}
\usepackage{threeparttablex}
\usepackage[normalem]{ulem}
\usepackage{makecell}
\usepackage{xcolor}
\ifLuaTeX
  \usepackage{selnolig}  % disable illegal ligatures
\fi
\usepackage[]{natbib}
\bibliographystyle{apalike}

\title{Multi-omic data science with R/Bioconductor}
\usepackage{etoolbox}
\makeatletter
\providecommand{\subtitle}[1]{% add subtitle to \maketitle
  \apptocmd{\@title}{\par {\large #1 \par}}{}{}
}
\makeatother
\subtitle{Welcome to Oulu Summer School, June 2022}
\author{}
\date{\vspace{-2.5em}2022-06-19}

\begin{document}
\maketitle

{
\setcounter{tocdepth}{1}
\tableofcontents
}
\hypertarget{overview}{%
\chapter{Overview}\label{overview}}

\hypertarget{contents-and-learning-goals}{%
\section{Contents and learning goals}\label{contents-and-learning-goals}}

This course will teach the \textbf{basics of biomedical data analysis with
R/Bioconductor}, a popular open source environment for scientific
data analysis. The participants get an overview of the reproducible
data analysis workflow in modern multi-omics, with a focus on recent
examples from published microbiome studies. After the course you will
know how to approach new tasks in biomedical data analysis by
utilizing available documentation and R tools.

The teaching will follow open online documentation created by the
course teachers, extending the online book Orchestrating Microbiome
Analysis (\url{https://microbiome.github.io/OMA}). The openly licensed
teaching material will be available online during and after the
course, following national recommendations on open education.

The training material walks you through the standard steps of
biomedical data analysis covering data access, exploration, analysis,
visualization, reproducible reporting, and best practices in open
science. We will teach generic data analytical skills that are
applicable to common data analysis tasks encountered in modern omics
research. The teaching format allows adaptations according to the
student's learning speed.

\hypertarget{schedule-and-organizers}{%
\section{Schedule and organizers}\label{schedule-and-organizers}}

The course will be organized in a live format (Flyer)

\textbf{Venue} University of Oulu. June 20-23, 2022.

\textbf{Schedule} Contact teaching daily between 9am -- 5pm, including
lectures, demonstrations, hands-on sessions, and breaks. A
detailed schedule is available at the course website:
(\url{https://microbiome.github.io/course_2022_oulu}).

\textbf{Teachers and organizers}

\href{https://datascience.utu.fi}{Leo Lahti} is the main teacher and Associate Professor in Data Science at the University of Turku, with specialization on biomedical data analysis. Course assistants are \emph{Tuomas Borman} (University of Turku) is one of the main developers of the open training material covered by the course and \emph{Jenni Hekkala}, a PhD researcher at the University of Oulu, in the group of the course coordinator Docent \emph{Justus Reunanen}.

The course is jointly organized by

\begin{itemize}
\tightlist
\item
  Health and Biosciences Doctoral Programme University of Oulu Graduate School
\item
  Cancer \& Translational Medicine Research Unit, University of Oulu
\item
  Department of Computing, University of Turku, Finland
\item
  Finnish IT Center for Science (CSC) supports the course with cloud
  computing services
\end{itemize}

\hypertarget{how-to-apply}{%
\section{How to apply}\label{how-to-apply}}

\textbf{Target audience}

The course is primarily designed for advanced MSc and PhD students,
Postdocs, and biomedical researchers who wish to learn and develop new skills in
scientific programming and biomedical data analysis. Academic students
and researchers from Finland and abroad are welcome and encouraged to
apply. The course has limited capacity of max 20 participants, and
priority will given for local students from Oulu.

\textbf{Expected background} Some earlier experience with R or another
programming language is recommended. However, this can be
compensated by familiarizing with the course material in advance, if
necessary. The teaching format allows adaptations according to the
student's learning speed.

\textbf{Application}

\begin{itemize}
\tightlist
\item
  Send a brief motivation letter to Jenni Hekkala \href{mailto:first.last@oulu.fi}{\nolinkurl{first.last@oulu.fi}}
\item
  Applications sent before May 20 will be given priority
\end{itemize}

\textbf{Course fee}

The course fee covers contact teaching and teaching material.

\begin{itemize}
\tightlist
\item
  285 euros with registration by May 20, 2022
\item
  350 euros with registration after May 20, 2022
\item
  Local students are exempted from the fee
\end{itemize}

\textbf{Accommodation}

Accommodation and travel costs are not included in the registration fee. For
accommodation tips, see \url{https://visitoulu.fi/en/arrival-overnight/}

\hypertarget{acknowledgments}{%
\section{Acknowledgments}\label{acknowledgments}}

\textbf{Citation} We thank all \href{https://microbiome.github.io}{developers and contributors} who have contributed open resources that supported the development of the training material. Kindly cite the course material as \citet{miacourse}

\textbf{Contact} See \url{https://microbiome.github.io}

\textbf{License and source code}

All material is released under the open \href{LICENSE}{CC BY-NC-SA 3.0 License} and available online during and after the course, following the \href{https://avointiede.fi/fi/linjaukset-ja-aineistot/kotimaiset-linjaukset/oppimisen-ja-oppimateriaalien-avoimuuden-linjaus}{recommendations on open teaching materials} of the national open science coordination in Finland**.

The source code of this repository is reproducible and contains
the Rmd files with executable code. All files can be rendered at one
go by running the file \url{main.R}. You can check the file for
details on how to clone the repository and convert it into a gitbook,
although this is not necessary for the training.

\begin{itemize}
\tightlist
\item
  Source code (github): \href{https://github.com/microbiome/course_2022_oulu}{miaverse teaching material}
\item
  Course page (html): \href{https://microbiome.github.io/course_2022_oulu/}{miaverse teaching material}
\end{itemize}

\hypertarget{program}{%
\chapter{Program}\label{program}}

The course takes place daily from 9am -- 5pm (CEST), including
coffee and lunch breaks.

We expect that participants will prepare for the course in advance, see section
\ref{start}. Online support is available.

The material follows open online book created by the course teachers,
Orchestrating Microbiome Analysis
\url{https://microbiome.github.io/OMA}. This is R/Bioconductor framework for
multi-omic data science.

Figure source: Moreno-Indias \emph{et al}. (2021) \href{https://doi.org/10.3389/fmicb.2021.635781}{Statistical and Machine Learning Techniques in Human Microbiome Studies: Contemporary Challenges and Solutions}. Frontiers in Microbiology 12:11.

\hypertarget{day-1---open-data-science}{%
\section{Day 1 - Open data science}\label{day-1---open-data-science}}

\textbf{Morning session}

9-10 Coffee, Welcome \& Practicalities

10-11 Lecture: Open \& reproducible workflows

11-12 Demo \& hands-on: Introduction to CSC RStudio notebook

12-13 Lunch break

\textbf{Afternoon hands-on session}

13-15 Demo: Data science framework

15-17 Hands-on: microbiome data summaries \& exploration

17-18 Presentations \& Discussion

\begin{center}\rule{0.5\linewidth}{0.5pt}\end{center}

\hypertarget{day-2---tabular-data}{%
\section{Day 2 - Tabular data}\label{day-2---tabular-data}}

\textbf{Morning session}

9-10 Lecture: Analysis \& visualization of \emph{tabular data}

10-12 Demo \& hands-on: Univariate methods

12-13 Lunch break

\textbf{Afternoon hands-on session}

13-14 Demo: Multivariate data analysis \& visualization

14-17 Hands-on: Multivariate data analysis \& visualization

17-18 Presentations \& Discussion

\begin{center}\rule{0.5\linewidth}{0.5pt}\end{center}

\hypertarget{day-3---multi-assay-data}{%
\section{Day 3 - Multi-assay data}\label{day-3---multi-assay-data}}

\textbf{Morning session}

9-10 Lecture: multi-omic data integration

10-12 Demo \& hands-on: multi-assay data container

12-13 Lunch break

\textbf{Afternoon hands-on session}

13-15: Demo \& hands-on: association analysis

13-17: Demo \& hands-on: machine learning

17-18 Presentations \& Discussion

\begin{center}\rule{0.5\linewidth}{0.5pt}\end{center}

\hypertarget{day-4---advanced-topics}{%
\section{Day 4 - Advanced topics}\label{day-4---advanced-topics}}

\textbf{Morning session}

9-10 Summary of the learning material

10-12 Demo \& hands-on: custom data \& advanced tools

12-13 Q \& A session

\textbf{Afternoon session}

13-14 Lunch

14-16 Wrap-up

\hypertarget{start}{%
\chapter{Getting started}\label{start}}

\hypertarget{checklist-before-the-course}{%
\section{Checklist (before the course)}\label{checklist-before-the-course}}

\hypertarget{csc-notebook}{%
\subsection{CSC Notebook}\label{csc-notebook}}

We will provide a temporary access to a cloud computing environment
that readily contains the available software packages. Instructions to access
the environment will be sent to the registered participants.

\begin{enumerate}
\def\labelenumi{\arabic{enumi}.}
\tightlist
\item
  Read the \href{https://docs.csc.fi/cloud/csc_notebooks/guide_for_students/}{instructions}
\item
  Go to the \href{https://notebooks-beta.rahtiapp.fi/welcome}{CSC notebook frontpage}
\item
  Login

  \begin{enumerate}
  \def\labelenumii{\alph{enumii}.}
  \tightlist
  \item
    Haka login

    \begin{itemize}
    \tightlist
    \item
      If you have a Finnish university account, you should be able to login with Haka
    \end{itemize}

    \begin{enumerate}
    \def\labelenumiii{\arabic{enumiii}.}
    \tightlist
    \item
      Press \textbf{Login} button from the frontpage
    \item
      Press \textbf{Haka} button
    \item
      Select right organization
    \item
      Enter login information
    \end{enumerate}
  \item
    CSC login

    \begin{itemize}
    \tightlist
    \item
      You can create a CSC account by following the \href{https://research.csc.fi/accounts-and-projects}{instructions}
    \end{itemize}

    \begin{enumerate}
    \def\labelenumiii{\arabic{enumiii}.}
    \tightlist
    \item
      Press \textbf{Login} button from the frontpage
    \item
      Press \textbf{CSC} button
    \item
      Enter login information
    \end{enumerate}
  \item
    Special login

    \begin{itemize}
    \tightlist
    \item
      For those who cannot login with Haka or CSC account
    \end{itemize}

    \begin{enumerate}
    \def\labelenumiii{\arabic{enumiii}.}
    \tightlist
    \item
      Contact Tuomas by email (\href{mailto:first.v.last@utu.fi}{\nolinkurl{first.v.last@utu.fi}}) if you are not able to login
    \item
      We give you a guest account
    \item
      Press \textbf{Special Login} button from the frontpage (below the \textbf{Login} button)
    \item
      Enter login information (username goes to \textbf{email} slot)
    \end{enumerate}
  \end{enumerate}
\item
  Join workspace

  \begin{enumerate}
  \def\labelenumii{\alph{enumii}.}
  \tightlist
  \item
    Press \textbf{Join workspace} button (Top right corner)
  \item
    Enter the \textbf{Join Code} (Check your email)
  \end{enumerate}
\item
  Start session

  \begin{enumerate}
  \def\labelenumii{\alph{enumii}.}
  \tightlist
  \item
    Press \textbf{ON} button
  \end{enumerate}
\item
  You can save files to \textbf{my-work} directory. They are kept stored even when the session is closed. \textbf{shared} folder is shared with all participants.
\end{enumerate}

\hypertarget{your-own-computer}{%
\subsection{(Your own computer)}\label{your-own-computer}}

Setting up the system on your own computer is not required for the
course but it can be useful for later use. The required software:

\begin{itemize}
\item
  \href{https://www.r-project.org/}{R (version \textgreater4.1.0)}
\item
  \href{https://www.rstudio.com/products/rstudio/download/}{RStudio};
  choose ``Rstudio Desktop'' to download the latest version. Optional
  but preferred. For further details, check the \href{https://www.rstudio.com/}{Rstudio home
  page}.
\item
  Install and load the required R packages (see Section \ref{packages})
\item
  After a successful installation you can start with the
  case study examples in this training material
\end{itemize}

\hypertarget{support-and-resources}{%
\section{Support and resources}\label{support-and-resources}}

\begin{itemize}
\tightlist
\item
  We recommend to have a look at the additional reading tips and try out online material listed in Section \ref{material}.
\end{itemize}

\textbf{You can run the workflows by simply copy-pasting the examples.} For
further, advanced material, you can test and modify further examples
from the online book, and apply these techniques to your own data.

\begin{itemize}
\tightlist
\item
  Online support on installation and other matters, join us at \href{https://gitter.im/microbiome/miaverse?utm_source=badge\&utm_medium=badge\&utm_campaign=pr-badge\&utm_content=badge}{Gitter}
\end{itemize}

\hypertarget{packages}{%
\section{Installing and loading the required R packages}\label{packages}}

Note that the CSC/RStudio environment has readily installed setup. You
may need the examples from this subsection if you are installing the
environment on your own computer. If you need to add new packages, you
can modify the examples below.

This section shows how to install and load all required packages into
the R session, if needed. Only uninstalled packages are installed.

\begin{Shaded}
\begin{Highlighting}[]
\CommentTok{\# List of packages that we need from cran and bioc }
\NormalTok{cran\_pkg }\OtherTok{\textless{}{-}} \FunctionTok{c}\NormalTok{(}\StringTok{"BiocManager"}\NormalTok{, }\StringTok{"bookdown"}\NormalTok{, }\StringTok{"dplyr"}\NormalTok{, }\StringTok{"ecodist"}\NormalTok{, }\StringTok{"ggplot2"}\NormalTok{, }
              \StringTok{"gridExtra"}\NormalTok{, }\StringTok{"kableExtra"}\NormalTok{,  }\StringTok{"knitr"}\NormalTok{, }\StringTok{"scales"}\NormalTok{, }\StringTok{"vegan"}\NormalTok{, }\StringTok{"matrixStats"}\NormalTok{)}
\NormalTok{bioc\_pkg }\OtherTok{\textless{}{-}} \FunctionTok{c}\NormalTok{(}\StringTok{"yulab.utils"}\NormalTok{,}\StringTok{"ggtree"}\NormalTok{,}\StringTok{"ANCOMBC"}\NormalTok{, }\StringTok{"ape"}\NormalTok{, }\StringTok{"DESeq2"}\NormalTok{, }\StringTok{"DirichletMultinomial"}\NormalTok{, }\StringTok{"mia"}\NormalTok{, }\StringTok{"miaViz"}\NormalTok{)}

\CommentTok{\# Get those packages that are already installed}
\NormalTok{cran\_pkg\_already\_installed }\OtherTok{\textless{}{-}}\NormalTok{ cran\_pkg[ cran\_pkg }\SpecialCharTok{\%in\%} \FunctionTok{installed.packages}\NormalTok{() ]}
\NormalTok{bioc\_pkg\_already\_installed }\OtherTok{\textless{}{-}}\NormalTok{ bioc\_pkg[ bioc\_pkg }\SpecialCharTok{\%in\%} \FunctionTok{installed.packages}\NormalTok{() ]}

\CommentTok{\# Get those packages that need to be installed}
\NormalTok{cran\_pkg\_to\_be\_installed }\OtherTok{\textless{}{-}} \FunctionTok{setdiff}\NormalTok{(cran\_pkg, cran\_pkg\_already\_installed)}
\NormalTok{bioc\_pkg\_to\_be\_installed }\OtherTok{\textless{}{-}} \FunctionTok{setdiff}\NormalTok{(bioc\_pkg, bioc\_pkg\_already\_installed)}

\CommentTok{\# Reorders bioc packages, so that mia and miaViz are first}
\NormalTok{bioc\_pkg }\OtherTok{\textless{}{-}} \FunctionTok{c}\NormalTok{(bioc\_pkg[ bioc\_pkg }\SpecialCharTok{\%in\%} \FunctionTok{c}\NormalTok{(}\StringTok{"mia"}\NormalTok{, }\StringTok{"miaViz"}\NormalTok{) ], }
\NormalTok{              bioc\_pkg[ }\SpecialCharTok{!}\NormalTok{bioc\_pkg }\SpecialCharTok{\%in\%} \FunctionTok{c}\NormalTok{(}\StringTok{"mia"}\NormalTok{, }\StringTok{"miaViz"}\NormalTok{) ] ) }

\CommentTok{\# Combine to one vector}
\NormalTok{packages }\OtherTok{\textless{}{-}} \FunctionTok{c}\NormalTok{(bioc\_pkg, cran\_pkg)}
\NormalTok{packages\_to\_install }\OtherTok{\textless{}{-}} \FunctionTok{c}\NormalTok{( bioc\_pkg\_to\_be\_installed, cran\_pkg\_to\_be\_installed )}
\end{Highlighting}
\end{Shaded}

\begin{Shaded}
\begin{Highlighting}[]
\CommentTok{\# If there are packages that need to be installed, install them }
\ControlFlowTok{if}\NormalTok{( }\FunctionTok{length}\NormalTok{(packages\_to\_install) ) \{}
\NormalTok{   BiocManager}\SpecialCharTok{::}\FunctionTok{install}\NormalTok{(packages\_to\_install)}
\NormalTok{\}}
\end{Highlighting}
\end{Shaded}

Now all required packages are installed, so let's load them into the session.
Some function names occur in multiple packages. That is why miaverse's packages
mia and miaViz are prioritized. Packages that are loaded first have higher priority.

\begin{Shaded}
\begin{Highlighting}[]
\CommentTok{\# Loading all packages into session. Returns true if package was successfully loaded.}
\NormalTok{loaded }\OtherTok{\textless{}{-}} \FunctionTok{sapply}\NormalTok{(packages, require, }\AttributeTok{character.only =} \ConstantTok{TRUE}\NormalTok{)}
\FunctionTok{as.data.frame}\NormalTok{(loaded)}
\end{Highlighting}
\end{Shaded}

\begin{verbatim}
##                      loaded
## mia                    TRUE
## miaViz                 TRUE
## yulab.utils            TRUE
## ggtree                 TRUE
## ANCOMBC                TRUE
## ape                    TRUE
## DESeq2                 TRUE
## DirichletMultinomial   TRUE
## BiocManager            TRUE
## bookdown               TRUE
## dplyr                  TRUE
## ecodist                TRUE
## ggplot2                TRUE
## gridExtra              TRUE
## kableExtra             TRUE
## knitr                  TRUE
## scales                 TRUE
## vegan                  TRUE
## matrixStats            TRUE
\end{verbatim}

\hypertarget{reproducible-reporting-with-rmarkdown}{%
\chapter{Reproducible reporting with Rmarkdown}\label{reproducible-reporting-with-rmarkdown}}

Reproducible reporting is the starting point for robust interactive
data science. Perform the following tasks:

\begin{itemize}
\item
  If you are entirely new to Markdown, take
  \href{https://www.markdowntutorial.com/}{this} 10 minute tutorial to get
  introduced to the most important functions within Markdown. Then
  experiment with different options with
  \href{https://www.rstudio.com/wp-content/uploads/2015/02/rmarkdown-cheatsheet.pdf}{Rmarkdown}
\item
  Create a Rmarkdown template in RStudio, and render it into a
  document (markdown, PDF, docx or other format). In case you are new
  to Rmarkdown \href{https://rmarkdown.rstudio.com/lesson-1.html}{Rstudio provides
  resources} to learn
  about the use cases and the basics of Rmarkdown.
\item
  Further examples are tips for Rmarkdown are available in the
  online tutorial to reproducible reporting by \href{https://rpubs.com/marschmi/RMarkdown}{Dr.~C Titus
  Brown}.
\end{itemize}

\hypertarget{material}{%
\chapter{Study material}\label{material}}

\hypertarget{online-tutorial}{%
\section{Online tutorial}\label{online-tutorial}}

The course will utilize material from the online book (beta version)
\href{https://microbiome.github.io/OMA/}{Orchestrating Microbiome Analysis with R/Bioconductor (OMA)}.

We encourage to familiarize with the material and test examples
already before the course.

\hypertarget{lecture-slides}{%
\section{Lecture slides}\label{lecture-slides}}

\href{https://github.com/microbiome/course_2022_oulu/tree/main/slides}{Slides} (will be added).

\hypertarget{tasks}{%
\section{Tasks}\label{tasks}}

\href{https://microbiome.github.io/OMA/exercises.html}{Exercises}

\hypertarget{extra-material-on-miaverse-and-r-programming}{%
\section{Extra material on miaverse and R programming}\label{extra-material-on-miaverse-and-r-programming}}

\href{https://microbiome.github.io/OMA/resources.html}{Resources}

In this course, we will analyze \href{https://rdrr.io/github/microbiome/microbiomeDataSets/man/HintikkaXOData.html}{HintikkaXOData}.
In this rat study, it was analyzed whether fats and prebiotics affects the microbiome.

The data consist of 4 groups:
- High-fat diet without prebiotics
- High-fat diet with prebiotics
- Low-fat diet without prebiotics
- Low-fat diet with prebiotics

You can find the data from \href{https://github.com/microbiome/course_2022_oulu/tree/main/data}{here}.

  \bibliography{packages.bib}

\end{document}
